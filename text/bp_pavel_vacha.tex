\documentclass[FM,BP,fonts]{tulthesis}

\newcommand{\verze}{2.0}

\usepackage{polyglossia}
\setdefaultlanguage{czech}
\usepackage{xevlna}

\usepackage{makeidx}
\makeindex

\usepackage{xunicode}
\usepackage{xltxtra}

% příkazy specifické pro tento dokument
\newcommand{\argument}[1]{{\ttfamily\color{\tulcolor}#1}}
\newcommand{\argumentindex}[1]{\argument{#1}\index{#1}}
\newcommand{\prostredi}[1]{\argumentindex{#1}}
\newcommand{\prikazneindex}[1]{\argument{\textbackslash #1}}
\newcommand{\prikaz}[1]{\prikazneindex{#1}\index{#1@\textbackslash #1}}
\newenvironment{myquote}{\begin{list}{}{\setlength\leftmargin\parindent}\item[]}{\end{list}}
\newenvironment{listing}{\begin{myquote}\color{\tulcolor}}{\end{myquote}}
\sloppy

% deklarace pro titulní stránku
\TULtitle{Predikce profilů spotřeby elektrické energie}{Prediction of power consumption profiles}
\TULauthor{Pavel Vácha}

% pro bakalářské, diplomové a disertační práce
\TULprogramme{B0613A140005}{Informační technologie}{Information Technology}
\TULbranch{B0613A140005AI-80}{Aplikovaná informatika}{Applied informatics}
\TULsupervisor{Ing. Jan Kraus, Ph.D.}

\TULyear{2024}



\begin{document}

\ThesisStart{male}
%\ThesisStart{zadani-a-prohlaseni.pdf}

\begin{abstractCZ}
Tato zpráva popisuje třídu \texttt{tulthesis} pro sazbu absolventských prací
Technické univerzity v~Liberci pomocí typografického systému \LaTeX.
\end{abstractCZ}

\begin{keywordsCZ}
spotřeba energie, analýza časových řad, predikce, parametrické modelování, strojové účení

\end{keywordsCZ}

\vspace{2cm}

\begin{abstractEN}
This report describes the \texttt{tulthesis} package for Technical university of
Liberec thesis typesetting using the \LaTeX\ typographic system.
\end{abstractEN}

\begin{keywordsEN}
energy consumption, time series analysis, forecasting, parametric modeling, machine learning
\end{keywordsEN}

\clearpage

\begin{acknowledgement}
Rád bych poděkoval všem, kteří přispěli ke vzniku tohoto díla. Zejména společnosti Albistech s.r.o za poskytnutá data a zázemí pro vypracování práce.
\end{acknowledgement}

\tableofcontents

\clearpage

\begin{abbrList}
\textbf{LSTM} & Long-short term memory, architektura rekurentní neuronové sítě \\
\textbf{CNN} & Convolutional Neural Network, konvoluční neuronová síť \\
\textbf{MSE} & Mean squared error, střední kvadratický chyba \\
\textbf{MAE} & Mean average error, průměrná absolutní odchylka \\
\textbf{MAPE} & Mean average error, průměrná procentuální absolutní odchylka \\
\textbf{RMSE} & Root mean squared error, směrodatný odchylka \\
\textbf{ReLU} & Rectifed Linear Unit, aktivační funkce \\
\textbf{GBT} & Gradient boosted trees, gradientní boostované stromy \\
\textbf{EDA} & Exploratory data analysis, explorační datová analýza

\end{abbrList}

\chapter{Úvod}

Vzorce spotřeby a jejich dopady na naše životní prostředí jsou aktuálně mezi největ-
šími výzvami naší doby. Díky pochopení jak a kdy lidé spotřebovávají energii a jak

se tyto vzorce mění, jsme schopni zajistit udržitelnou budoucnost.
Tradiční modely spotřeby se obvykle opírají o národní nebo globální data, která

nemusí zcela odrážet lokální vzorce spotřeby nebo jejich podmínky (např. klimatic-
ké). Naproti tomu modely lokální spotřeby mohou poskytnout přesnější a mnohem

relevantnější informace pro konkrétní region či oblast.
V tomto projektu si kladu za cíl vyvinout model lokální spotřeby založený na
historických datech spolu s environmentálními parametry, pokud jsou pro danou
oblast dostupné. Tento model může být velmi cenný pro plánování a implementaci
udržitelných opatření v oblasti energetiky a ochrany životního prostředí. Získání
přesnějšího a místně relevantního pohledu na spotřebu energie nám umožní přijímat
informovaná rozhodnutí a přizpůsobit naše strategie tak, aby byly co nejefektivnější
a nejohleduplnější k životnímu prostředí.
Pro vytvoření spolehlivého a vhodného modelu lokální spotřeby se část tohoto
projektu bude zabývat rešerší různých technik strojového učení spolu s metodami
ze statistické analýzy.



\end{document}
